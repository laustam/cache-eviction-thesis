% An abstract is a brief, self-contained summary of your entire thesis. Its purpose is to give the reader a quick overview of your research so they can decide if they want to read the full paper. For a thesis, it's typically between 150-300 words.

% A strong abstract follows a specific four-part structure:

% 1. Context and Problem Statement
% Start by setting the scene. Why is this research important? What is the problem or gap in knowledge you are addressing?

% Example: "The performance of modern web systems relies heavily on the efficiency of their caching mechanisms. However, the choice of an optimal cache eviction algorithm for real-world workloads, which often follow a Zipfian distribution, remains a critical challenge."

% 2. Methodology
% Briefly explain how you conducted your research. What did you do to solve the problem?

% Example: "This thesis evaluates the performance of three cache eviction primitives—SIEVE, LRU, and FIFO—through a series of controlled simulations. Performance was measured using the miss ratio of each algorithm across a range of cache sizes and varying degrees of Zipfian skewness."

% 3. Key Findings
% Summarize your most important results. What were the main takeaways from your data?

% Example: "The findings demonstrate that SIEVE consistently outperforms both LRU and FIFO. This performance advantage is robust across all tested parameters, becoming particularly pronounced as the workload's skewness increases. SIEVE's miss ratio distribution was found to be significantly lower by every key statistical metric, including mean, median, and interquartile range."

% 4. Conclusion and Implications
% Conclude by stating the significance of your findings. What does your research contribute to the field?

% Example: "This work confirms SIEVE's viability as a superior alternative to established cache policies for Zipf-like workloads. It provides a strong theoretical framework and benchmark for future research into cache eviction, particularly in systems where access patterns are highly skewed."

% By following this structure, you can create a clear and impactful abstract for your thesis.

\providecommand{\keywords}[1]
{
  \small
  \textbf{\textit{Keywords---}} #1
}

\begin{abstract}

The performance of modern web systems relies heavily on the efficiency of their caching mechanisms. However, the choice of an optimal cache eviction algorithm for real-world workloads, which often follow a Zipf-like distribution, remains a critical challenge. This thesis evaluates the performance of three cache eviction primitives (SIEVE, LRU, and FIFO) through a series of simulations on workloads generated by sampling from Zipfian distributions. Performance was measured using the miss ratio of each algorithm across varying degrees of Zipfian skewness, cache sizes, and working set sizes. The findings demonstrate that SIEVE consistently outperforms both LRU and FIFO. This performance advantage is robust across multiple tested parameters, becoming particularly pronounced as the workload's skewness increases. This work confirms SIEVE's viability as an alternative to established cache policies for Zipf-like workloads. It provides a strong theoretical framework and benchmark for future research into cache eviction, particularly in systems where access patterns are highly skewed.

\vspace{1em}
\keywords{
    cache eviction primitives,
    cache eviction,
    SIEVE,
    LRU,
    FIFO,
    miss ratio,
    Zipfian distribution,
    web caching
}

\vfill

\begin{center}
    \textit{All code used for this project can be found on GitHub, at} \url{https://github.com/laustam/cache-eviction-thesis}
\end{center}

\end{abstract}



% This work explored the performance of different cache eviction primitives (SIEVE, LRU, and FIFO) using synthetic workloads drawn from Zipf-like distributions. The primary objective was to gain insight into whether SIEVE's performance is comparable to that of established LRU and FIFO policies. The findings demonstrate a strong preference for SIEVE over its competitors, with its miss ratio consistently being lower across varying Zipfian skewness parameters ($\alpha$) and relative cache sizes. This research provides valuable insights into algorithm-independent cache miss ratios and establishes a strong theoretical framework for more extensive future work.
